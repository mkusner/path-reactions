% !TEX root =  ../main.tex

\subsection{Data and preprocessing}

Our dataset for evaluating our model is a collection of chemical reactions extracted from the US patent database\footnote{https://bitbucket.org/dan2097/patent-reaction-extraction/downloads/} \todo{not sure how to cite this}.
We take as our starting point the 479,035 reactions, along with the training, validation, and testing splits, 
which were used by \citet{jin2017predicting}, referred to as the USPTO dataset.
This data is in the form of reaction SMILES strings, a text-based format used for encoding the molecules present in the reaction.
Before applying our method, we perform two data preprocessing tasks in order to extract a subset of data appropriate for training a model of electron movement during a reaction.

\subsubsection{Reactant and reagent separation}

Typically, reaction SMILES strings are split into three parts --- reactants, reagents, and products.
The {\em reactant} molecules are those which are consumed during the course of the chemical reaction to form the {\em product}, 
while the {\em reagents} are any additional molecules which provide context under which the reaction occurs (for example, catalysts),
but do not explicitly take part in the reaction \todo{give an example here?} itself.
Each non-hydrogen atom in the USPTO dataset is assigned an {\em atom map} number, an integer which serves as an unique identifier for the atom and allows us to match particular atoms in the reactants with atoms in the products.

Unfortunately, the USPTO dataset as extracted does not differentiate between reagents and reactants.
We elect to preprocess the entire USPTO dataset by separating out the reagents from the reactants using the process outlined in \citet{schwaller2017found}, where we classify as a reagent any molecule for which either 
(i) none of its constituent atoms appear in the product, or 
(ii) the molecule appears in the product SMILES completely unchanged from the pre-reaction SMILES.
This allows us to properly model molecules which are included in the dataset but do not materially contribute to the reaction.

\subsubsection{Identifying elementary heterolytic reactions}

To train our model, it is necessary to extract a ground-truth representation of the electron paths from the SMILES strings.
Furthermore, not every reaction in the USPTO dataset is an elementary heterolytic reaction; 
such reactions (for example, multi-step reactions) will not have a single unique path through the atoms 
which describes the movement of the electrons.

Extracting the paths and filtering the dataset is a several-step process.
The first step is to construct adjacency matrices for the reactants and for the product,
and compare the number of bonds between each pair of atoms.
To do this in a consistent way, we first read the SMILES string for the molecules in the reaction
into the open-source chemoinformatics software RDKit.
We then place the molecule in Kekule form, a process which makes explicit the location of single and double bonds in aromatic structures.
Given the adjacency matrices $A_r$ for the reactants and $A_p$ for the products,
where the entries denote the number of bonds,
computing the difference $\Delta = A_p - A_r$ gives a matrix of all bond changes during the course of the reaction;
positive entries denote new bonds that are formed, and negative entries denote bonds that break.
If there are any entries in $\Delta$ which are not -1, 0, or +1, then the reaction is a multi-step reaction
which contains more than one electron path, and we filter it out for training purposes.

For the reactions that remain, we then extract an electron path from the difference matrix $\Delta$.
Each path is defined as a sequence of pairs of atoms, alternating between add steps (+1) and remove steps (-1).
%which can be extracted by  any atom that has a non-zero entry in its row 
To correct for any issues that caused due to reactant and product molecules being kekulized in different ways,
we first check if there is any through the graph which forms a cycle, 
\todo{what's the deal with this and pericyclic reactions?}{}
and set all entries in $\Delta$ corresponding to this cycle to 0.
The remaining non-zero entries in $\Delta$ fully describe a unique path through the atoms,
which we extract by starting at any atom which has exactly one non-zero entry in its row,
and then following the adjacency matrix.
To determine the direction of this path, we look at the change in the charge of the atoms
between the products and the reactants:
the atom at the end of the path will have a more negative charge in the product than in the reactants,
and vice-versa. \todo{JOHN what do we actually do in the code here?}{}

The end result of this is extracted reaction paths for those entries in the USPTO dataset which 
correspond to elementary heterolytic reactions.
This comprises the majority of the dataset, containing 349,898 total reactions, of which 29,360 form the held-out test set.


