
\paragraph{LEF Topology}
\added[id=jab]{
\ourModel can currently only predict reactions with LEF topology (\S \ref{sect:LEF}). 
These are the most common form of reactions \citep{herges1994organizing}, but in future work we would like to extend \ourModel's action repertoire to work with other types of mechanisms such as cyclic reactions.
 This could be done by allowing \ourModel to output multiple paths as well as providing it with labelled mechanism paths for these reactions, obtainable from finer grained datasets. }
 
 \paragraph{Graph representation of molecules}
 \added[id=jab]{
 Although this shortcoming is not just restricted to our work, by modelling molecules and reactions as graphs and operations thereon, we ignore details about the electronic structure, and conformational information, ie information about how the molecule shape changes in 3D. 
 This information is crucial in some cases.
 Having said this, there is probably some balance to be struck here, as representing molecules and reactions as graphs is an extremely powerful abstraction, and one that is commonly used by chemists, aiding it's interoperability.}
 
 
