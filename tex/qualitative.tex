% !TEX root =  ../main.tex

\begin{figure*}[t]

    \centering
    \begin{subfigure}[b]{0.3\textwidth}
        \centering
        \includegraphics[height=0.9in]{imgs/textbook/reaction3}\\\vspace{0.1in}
        %\caption{}
    \end{subfigure}%
    \hspace{1cm}
     \begin{subfigure}[b]{0.5\textwidth}
        \centering
        \includegraphics[height=1.1in]{imgs/textbook/reaction7}
        %\caption{}
    \end{subfigure}
%    \vspace{-1em}
	\caption{(left) 2nd-order nucleophilic substitutions $S_N 2$-reactions, (right) Suzuki-coupling (please note that in the "real" mechanism of the Suzuki coupling, the reaction would proceed via oxidative insertion, transmetallation and reductive elimination at a Palladium catalyst. In this work, we only condition implicitly on catalysts and reagents). In both cases, our model has correctly picked up the trend that halides lower in the period table usually react preferably ($I>Br>Cl$). }
	\label{fig:qualitative}
\vspace{-0.5em}
\end{figure*}



\subsection{Qualitative Analysis}

Complex molecules often feature several potentially reactive functional groups $r=\{F_1,...,F_N\}$, which compete for reaction partners. 
To predict the selectivity, that is which functional group will predominantly react in the presence of other groups, 
students of chemistry learn heuristics and trends, 
which have been established over the course of three centuries of experimental observation.
To qualitatively study whether the model has learned such trends from data we queried the model with several typical text book examples from the chemical curriculum (see Figure \ref{fig:qualitative} and the Appendix). 
We found that the model predicts most examples correctly. In the few incorrect cases, interpreting the model's output reveals that the model made chemically plausible predictions.

