<<<<<<< HEAD
We start by summarizing the related work on \emph{product prediction} and \emph{mechanism prediction} with a particular emphasis on whether the techniques are end-to-end trainable. Table~\ref{table.existing} gives a summary of this section.


\begin{wrapfigure}{R}{0.6\textwidth}
\vspace{-4ex}
\begin{minipage}{0.6\textwidth}
\begin{table}[H]
\begin{tabular}{c|ccc} 
\hline
 \textbf{Prior Work} & \textbf{end-to-end} & \textbf{mechanism}  \\ \hline \hline
\cite{wei2016neural} & \checkmark & $-$  \\ \hline
\cite{coley2017prediction} & $-$ & $-$ \\ \hline
\cite{jin2017predicting} &$-$ &$-$ &  \\ \hline
\cite{schwaller2017found} & \checkmark &$-$  \\ \hline
\cite{segler2017modelling} & \checkmark & $-$ \\ \hline 
\cite{segler2018planning} & \checkmark &$-$  \\ \hline
\cite{NIPS2011_4356} &$-$ & \checkmark  \\ \hline
\cite{kayala2011learning} & $-$ & \checkmark \\ \hline
\cite{kayala2012reactionpredictor} &$-$ & \checkmark  \\ \hline
\cite{fooshee2018deep} & $-$ & \checkmark \\ \hline
\textbf{this work} & \checkmark & \checkmark \\
\hline
%\bf{Method} & %\multicolumn{2}{c}{\bf Full} & \multicolumn{2}{c}{\bf Unaware} & \multicolumn{2}{c}{\bf Fair L2} & \multicolumn{2}{c}{\bf Fair L3} \\
\end{tabular}
\centering
	\caption{Work on machine learning for reaction prediction, and whether they are (a) end-to-end trainable and (b) predict the reaction mechanism. \label{table.existing}}
\end{table}
\end{minipage}
\vspace{-4ex}
\end{wrapfigure}


\paragraph{Product prediction.}

% wei2016neural
% NOTE: Jennifer, David and Alan were not the first to apply deep learning to this problem, this was done Aires de Sousa's group already in 2005, using unsupervised pretraining.
Recently, methods combining machine learning and global rewriting rules have been proposed [\cite{coley2017prediction,neural-symbolic,segler2018planning,wei2016neural,zhang2005structure}]. Here, a learned model is used to predict which rewrite rule to apply preferably. While these models are readily interpretable, they tend be brittle. 
%The earliest work we are aware of that uses deep learning to predict the products of reactions is \cite{zhang2005structure}. Their idea was to approximate the operations of a molecular fingerprint so that all operations become continuously differentiable. They then learned parameters of this fingerprint to accurately predict product fingerprints end-to-end.

% jin2017predicting
Another class of models is the graph-based technique by \cite{jin2017predicting} who construct a network based on the Weisfeiler-Lehman algorithm for testing graph isomorphism. They use this algorithm to select atoms that will be involved in a reaction. They then enumerate all chemically-valid bond changes involving these atoms and learn a separate network to rank the resulting potential products. This method is state-of-the-art on product prediction.

% schwaller2017found
\cite{schwaller2017found} represents reactants as SMILES (CITE) strings and then used a sequence to sequence network (specifically, the work of (CITE)) to predict product SMILES. While this method is end-to-end trainable the SMILES representation is quite brittle as often single character changes will not correspond to a valid molecule.
% segler2017modelling
% TODO
% segler2018planning
% TODO

\paragraph{Mechanism prediction.}
The only other work we are aware of to use machine learning to predict reaction mechanisms are \cite{fooshee2018deep,kayala2012reactionpredictor,NIPS2011_4356,kayala2011learning}.
All of these model a chemical reaction as an interaction between atoms which function as electron donors and those which function as electron acceptors. They predict the reaction mechanism via two independent models: one that identifies these likely electron sources and sinks, and another that ranks all combinations of them.
However, this combining and then ranking of electron sources and sinks can be a slow process, as many plausible reactions need to be considered (the number of subgraphs of $n$ reacting atoms, where single bonds are either added or removed is $2^{n(n-1)/2}$).
These models also have only so far been successfully trained on small hand-curated datasets.
% kayala NIPS: NIPS2011_4356
% kayala2011learning
% kayala2012reactionpredictor
% fooshee2018deep

=======
>>>>>>> master
