% !TEX root =  ../main.tex

\subsection{Results and Evaluation}

Having trained our model, one surprising challenge is accurately defining what it means for our model to ``correctly'' predict the reaction outcome. 
Part of this relates to whether we are interested in  \emph{Reaction Mechanism Prediction} or \emph{Reaction Product Prediction}. \todo{these should have been described by now using fig 1 but back check.} 
In this section we evaluate our model in each of these regimes in turn.
%Sampling from our model, or selecting high-probability candidates using beam search, yields

\subsubsection{Reaction Mechanism Prediction}

 For Reaction Mechanism Prediction we are interested in making sure that we got the exact sequence of actions correct.
For instance, when forming a bond between two pairs of atoms we want to know which one of the atoms donated the electron pair needed to form the bond, even if the end result is the same. 

The representation of the reaction mechanism produced by our model is a sequence of atoms, detailing the path taken by the electrons in a series of alternating steps in which bonds are broken and formed; this then can take the form of a sequence of integers.
Here in order to associate a fixed identity with each atom, these integers represent the ``atom mapped'' labels\footnote{In the USPTO dataset, all reactions have been ``atom mapped'', which means that integer labels have been assigned to each non-hydrogen atom in both the reactants and the products.} associated to each atom in the sequence.

The most straightforward approach then to evaluate our accuracy at predicting reaction mechanisms is to check whether the sequence of integers extracted from the raw data (by comparing the reported major product with the reactants\todo[]{change bracket content to "as described earlier"...?}) is an exact match with the sequence of integers output by our \ourModel; the top-1, top-2, top-3, and top-5 accuracies evaluated in this manner are reported in Table \ref{table:mech-predict}.

\begin{table}[h]
  \caption{Results when using \ourModel  for Reaction Mechanism Prediction. Here we count a prediction as correct if the atom mapped action sequences predicted by our model match exactly those extracted from the USPTO dataset.}
  \label{table:mech-predict}
  \centering
  \begin{tabular}{lllll}
    \toprule
    & \multicolumn{4}{c}{Accuracies (\%)}                   \\
    \cmidrule(r){2-5}
    Model Name & Top-1 & Top-2 & Top-3 & Top-5 \\
    \midrule
    \ourModelIR &  70.3 &  82.8 & 87.7 & 92.2    \\
    \ourModelR  &  77.8 &  89.2 & 92.4 & 94.7    \\
    \bottomrule
  \end{tabular}
\end{table}




\subsubsection{Reaction Product Prediction}
\label{sec:product-prediction}

Reaction Mechanism Prediction is useful for ensuring that we formed the correct product in the {\em correct way}.
However, this can underestimate the actual predictive accuracy of the model: 
although a single atom mapping is provided as part of the USPTO dataset, in general atom mappings are not unique; 
when a reactant contains some symmetries, then multiple different atom mappings are effectively equivalent.
Here this would manifest as multiple different sequences of integers which correspond to chemically identical electron paths. An example of a reaction falling under this category is given in the supplementary material.
\sout{An extreme example is a reaction in which one reactant is benzene, a molecule which is formed of six carbon atoms in a completely symmetric ring: while this would be present with a unique atom mapping in the dataset, any reaction path which includes one of these carbon atoms could equally well have selected any of the other five \todo{maybe toss in an inline figure with an example}.}

Recent machine learning approaches to Reaction Product Prediction \citep{jin2017predicting,schwaller2017found}
have evaluated whether the major product reported in the test dataset matches predicted candidate products generated by their system, independent of mechanism.
In our case, the top-5 accuracy for a particular reaction may include multiple different electron paths that ultimately yield the same product molecule.

Identifying whether two product molecules are chemically the same is equivalent to solving a graph isomorphism over the atoms and bond types, comparing the output of our system to the product molecule.
To perform this comparison, we consider an electron path as a sequence of edits performed on the reactants graph, apply these edits to define a product graph, 
and then define a deterministic mapping from the edited graph to a canonical string representation.
This is done by first Kekulizing the molecule, a process which makes explicit the location of single and double bonds in aromatic structures.
We then apply the sequence of edits to the reactants graph,
set explicit charges or hydrogen counts on the first and last atom in the electron path in order to satisfy valence constraints,
and strip all atom map numbers from the graph.
If this graph corresponds to a valid product molecule, we can then use RDKit to express the molecule in a canonical SMILES string format;
predicted electron paths which would yield chemically infeasible products are represented as an empty string.
We can then evaluate whether a predicted electron path matches the ground truth electron path by a string comparison.

To use our model to produce a ranked list of predicted products, we can compute the estimated canonicalized product SMILES for each of the outputs of our beam search over electron paths, removing duplicates along the way. 
These product-level accuracies are reported in Table~\ref{table:prod-predict}.
For product prediction we compare with the state-of-the-art graph-based method \cite{jin2017predicting};
we use their evaluation code and pre-trained model \footnote{\url{https://github.com/wengong-jin/nips17-rexgen}},
re-evaluated on our extracted set of elementary heterolytic reactions.
%We note their performance is slightly better on this test set than on the USPTO dataset as a whole.
We also trained a Seq2Seq baseline model following \cite{schwaller2017found}. 
Unfortunately, with the hyperparameter settings described in the paper we were not able to reproduce their results.
As the authors have not released their code or a trained model, we are unable to evaluate its performance on our test set, and instead quote their reported performance on a different USPTO test set for reference.
Overall, \ourModelR outperforms all other approaches on this metric, with 87\% top-1 accuracy and 95.9\% top-5 accuracy.
Omitting the reagents in \ourModelR degrades top-1 accuracy slightly, but maintains a high top-3 and top-5 accuracy,
suggesting that reagent information is necessary to provide context in disambiguating different plausible reaction paths;
we will explore this further below.



\begin{table}[h]
  \caption{Results when using \ourModel  for Reaction Mechanism Prediction. Here we consider a prediction correct following the product matching procedure in Section~\ref{sec:product-prediction}. 
The WLDN accuracy is computed using their own evaluation code and pretrained model outputs, on our test set.
We were unable to evaluate the Seq2Seq model on our test set, instead quoting their reported numbers with an asterisk.}
  \label{table:prod-predict}
  \centering
  \begin{tabular}{lllll}
    \toprule
    & \multicolumn{4}{c}{Accuracies (\%)}                   \\
    \cmidrule(r){2-5}
    Model Name & Top-1 & Top-2 & Top-3 & Top-5 \\
    \midrule
    \ourModelIR &  78.2 & 87.7 & 91.5 & 94.4   \\
    \ourModelR  &  {\bf 87.0} & {\bf 92.6} & {\bf 94.5} & {\bf 95.9}    \\
    \bottomrule \toprule
    WLDN \citep{jin2017predicting} & 84.0  & 89.2 &  91.1 & 92.3 \\
    Seq2Seq \citep{schwaller2017found} & 80.3$^\star$ & 84.7$^\star$ & 86.2$^\star$ & 87.5$^\star$ \\
    \bottomrule
  \end{tabular}
\end{table}



A benefit of our approach, even if the ultimate the desired goal is to predict the product molecule rather than the electron path,
is that the predicted electron paths then can serve as explanation.
For each predicted product molecule, there is at least one electron path generated by our model which produced this;
whichever of these was highest-ranked by the beam search corresponds to the maximum likelihood path, 
but we can also report the other candidate paths which would produce the sample output as alternative explanations.
\highlight{MAYBE EXAMPLE IN APPENDIX FIGURE.}



