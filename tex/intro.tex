% !TEX root =  ../main.tex
% reaction prediction is important
% currently it is an arduous process that requires a chemist
% cost
% human effort

The ability to reliably predict the products of chemical reactions is of tremendous importance for areas as diverse as health care, renewable energy, and construction, providing molecules which serve as medicines, energy capturing devices, and nanomaterials. 
When designing most new molecules, often a very experienced chemist with years of experience is needed, in order to make accurate predictions. 
If such predictions could be automated it could drastically speed up the discovery of new molecules for these applications, among many others. Recently, there have been a number of machine learning models proposed for predicting the product of chemical reactions \cite{coley2017prediction,jin2017predicting,neural-symbolic,schwaller2017found,wei2016neural,zhang2005structure}, largely using graph-based or machine translation models. 

Apart from reaction product prediction, another crucial aspect of chemical reactions is the \emph{reaction mechanism}. Theoretically, all chemical reactions can be described by the stepwise movement of electrons in molecules, called the reaction mechanism.
This can be treated at different levels of abstraction. On the lowest level, quantum-mechanical simulations of the  changes in electronic structure are calculated by approximately solving the Schrödinger equation, which is computationally expensive for most systems of interest. 
On the other end, chemical reactions can be treated as rules that ``rewrite'' reactant molecules to products, which abstracts away the individual electron redistribution steps into a single transformation step. 
%While rules can be brittle, they allow to organize chemical knowledge by grouping reactions by these rules, which facilitates learning.
To combine the advantages of both approaches, chemists use a tremendously powerful model, which simplifies the stepwise electron shifts using sequences of arrows which indicate bond making and breaking \cite{herges1994organizing}. 
%Using this compositional abstraction, chemists are able to make predictions beyond the established global rules, and understand the underlying mechanism, with essentially just pencil and paper.

Understanding the reaction mechanism is important because it not only determines the products but it allows one to understand the general trends inherent in reactions.



% Machine learning has been applied mostly for the quantum chemistry-level\cite{NIPS2012_4830,schutt2017schnet}. Additionally, various models have been proposed to predict global rewriting rules for reaction prediction\cite{coley2017prediction,jin2017predicting,neural-symbolic,schwaller2017found,wei2016neural,zhang2005structure}. These models can be trained on large sets of reported chemical reactions. 
At the medium level of abstraction, Kayala et al. proposed a model to predict electron shift steps using of two independently learned learning-to-rank and scoring stages,
 which however relies on complex hand-coded rules and expert-annotated datasets, which are usually very small.\cite{kayala2011learning,kayala2012reactionpredictor} 


Here, we propose ElectronNet, the first end-to-end model to learn sequences of electron shifts directly from large, unannotated reaction datasets \todo{wording not yet elegant here}.