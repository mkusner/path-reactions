% !TEX root =  ../main.tex

%The ability to reliably predict the products of chemical reactions is of tremendous importance for areas as diverse as health care, renewable energy, and construction, providing molecules which serve as medicines, energy capturing devices, and nanomaterials. 
%Assisting chemists by automating reaction predictions could drastically speed up the discovery of new molecules for these applications, among many others. Recently, there have been a number of machine learning models proposed for predicting the products of chemical reactions \cite{coley2017prediction,jin2017predicting,schwaller2017found,neural-symbolic,segler2018planning,wei2016neural}, largely using graph-based or machine translation models. The task of reaction product prediction is shown on the left-hand side of Figure~\ref{fig:task-overview}.


Theoretically, all chemical reactions can be described by the stepwise rearrangement of electrons in molecules \cite{herges1994organizing}. 
This sequence of bond-making and breaking is known as the \emph{reaction mechanism}. 
Understanding the reaction mechanism is crucial because it not only determines the products (formed at the last step of the mechanism), 
but it also provides insight into why the products are formed on an atomistic level. 
%
Mechanisms can be treated at different levels of abstraction. On the lowest level, quantum-mechanical simulations of the electronic structure can be performed, which is computationally expensive for most systems of interest. 
On the other end, chemical reactions can be treated as rules that ``rewrite'' reactant molecules to products, which abstracts away the individual electron redistribution steps into a single, global transformation step. 
To combine the advantages of both approaches, chemists use a tremendously powerful intermediate model, which simplifies the stepwise electron shifts using sequences of arrows which indicate the path of electrons throughout molecular graphs \cite{herges1994organizing}. 


Recently, there have been a number of machine learning models proposed for directly predicting the products of chemical reactions on a global level\cite{coley2017prediction,jin2017predicting,schwaller2017found,neural-symbolic,segler2018planning,wei2016neural}, largely using graph-based or machine translation models. The task of reaction product prediction is shown on the left-hand side of Figure~\ref{fig:task-overview}. On the other hand, approaches combining hand-coded heuristics and quantum-mechanics, but not machine learning, have been suggested to predict reaction mechanisms.\cite{kim2018efficient,rappoport2014complex,zimmerman2013automated,bergeler2015heuristics,simm2017context}


In this paper we propose a machine learning model to predict the reaction mechanism, as shown on the right-hand side of Figure~\ref{fig:task-overview}, of a particularly important subset of organic reactions.
We argue that not only is our model more interpretable than product prediction models but it is easier to encode the constraints imposed by chemistry into it. 
We call our model \ourModel, as it directly predicts the path of electrons through molecules (i.e., the reaction mechanism). To train the model we devise a general technique to obtain approximate reaction mechanisms purely from data about the reactants and products. This allows one to train our a model on large, unannotated reaction datasets such as USPTO \cite{lowe2012extraction}. We demonstrate that not only does our model achieve state-of-the-art results, it also surprisingly learns chemical properties it was not explicitly trained on.



%Here, we propose ElectronNet, the first end-to-end model to learn sequences of electron shifts directly from large, unannotated reaction datasets \todo{wording not yet elegant here}.