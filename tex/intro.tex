% !TEX root =  ../main.tex
% reaction prediction is important
% currently it is an arduous process that requires a chemist
% cost
% human effort

The ability to reliably predict the products of chemical reactions is of tremendous importance for areas as diverse as health care, renewable energy, and construction, providing molecules which serve as medicines, energy capturing devices, and nanomaterials. 
Assisting chemists by automating reaction predictions could drastically speed up the discovery of new molecules for these applications, among many others. Recently, there have been a number of machine learning models proposed for predicting the products of chemical reactions \cite{coley2017prediction,jin2017predicting,schwaller2017found,neural-symbolic,segler2018planning,wei2016neural}, largely using graph-based or machine translation models. The general task of reaction prediction is shown on the left-hand side of Figure~\ref{fig:task-overview}.

%Apart from reaction product prediction, another crucial aspect of chemical reactions is the \emph{reaction mechanism}. 
Theoretically, all chemical reactions can be described by the stepwise movement of electrons in molecules, called the reaction mechanism. The final reaction products are the molecules formed at the last step of the mechanism. 
Mechanisms can be treated at different levels of abstraction. On the lowest level, quantum-mechanical simulations of the electronic structure can be performed, which is computationally expensive for most systems of interest. 
On the other end, chemical reactions can be treated as rules that ``rewrite'' reactant molecules to products, which abstracts away the individual electron redistribution steps into a single transformation step. 
To combine the advantages of both approaches, chemists use a tremendously powerful intermediate model, which simplifies the stepwise electron shifts using sequences of arrows which indicate the path of electrons throughout molecular graphs \cite{herges1994organizing}. 
Understanding the reaction mechanism is crucial because it not only determines the products, but it also allows one to understand the general trends inherent in reactions. %This Further, many rules used by chemists to understand the product of complex reactions is based on the reaction mechanism.
%MATT: Marwin could you check me on this?


In this paper we propose a model to predict the reaction mechanism, as shown on the right-hand side of Figure~\ref{fig:task-overview}, of a particularly important subset of organic reactions.
We argue that not only is our model more interpretable than product prediction models but it is easier to encode the constraints imposed by chemistry into it. 
We call our model \ourModel, as it directly predicts the path of electrons through molecules (the reaction mechanism). To train the model we devise a general technique to obtain formal reaction mechanisms purely from data about the reactants and products of a reaction. This allows  the training of a model on large, unannotated reaction datasets such as USPTO \cite{lowe2012extraction}. 



%Here, we propose ElectronNet, the first end-to-end model to learn sequences of electron shifts directly from large, unannotated reaction datasets \todo{wording not yet elegant here}.