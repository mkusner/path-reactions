\documentclass{article}

% if you need to pass options to natbib, use, e.g.:
\PassOptionsToPackage{numbers, compress}{natbib}
% before loading nips_2017
%
% to avoid loading the natbib package, add option nonatbib:
% \usepackage[nonatbib]{nips_2017}

\usepackage{nips_2018}

% to compile a camera-ready version, add the [final] option, e.g.:
% \usepackage[final]{nips_2017}

\usepackage[utf8]{inputenc} % allow utf-8 input
\usepackage[T1]{fontenc}    % use 8-bit T1 fonts
\usepackage{hyperref}       % hyperlinks
\usepackage{url}            % simple URL typesetting
\usepackage{booktabs}       % professional-quality tables
\usepackage{amsfonts}       % blackboard math symbols
\usepackage{nicefrac}       % compact symbols for 1/2, etc.
\usepackage{microtype}      % microtypography

\usepackage{hyperref}

\usepackage{amssymb}
\usepackage{amsmath}

% For citations
\usepackage{natbib}

% For figures
\usepackage{graphicx} % more modern
\usepackage{wrapfig}
%\usepackage{epsfig} % less modern
\usepackage{subfigure} 
\usepackage{multirow}
\usepackage{adjustbox}

\usepackage{listings}
\usepackage{textcomp}

% For assumptions
\usepackage{amsthm,amssymb,amsopn}
\newtheorem{assumption}{Assumption}
\newtheorem{define}{Definition}
\newtheorem{thm}{Theorem}
\newtheorem{lem}{Lemma}
\newtheorem{coro}{Corollary}
\newtheorem{condition}{Condition}

% For algorithms
\usepackage{algorithm}
\usepackage{algorithmic}
\renewcommand{\algorithmicrequire}{\textbf{Input:}}
\renewcommand{\algorithmicensure}{\textbf{Output:}}
\makeatletter
\makeatletter
\newcommand*{\da@rightarrow}{\mathchar"0\hexnumber@\symAMSa 4B }
\newcommand*{\da@leftarrow}{\mathchar"0\hexnumber@\symAMSa 4C }
\newcommand*{\xdashrightarrow}[2][]{%
  \mathrel{%
    \mathpalette{\da@xarrow{#1}{#2}{}\da@rightarrow{\,}{}}{}%
  }%
}
\newcommand{\xdashleftarrow}[2][]{%
  \mathrel{%
    \mathpalette{\da@xarrow{#1}{#2}\da@leftarrow{}{}{\,}}{}%
  }%
}
\newcommand*{\da@xarrow}[7]{%
  % #1: below
  % #2: above
  % #3: arrow left
  % #4: arrow right
  % #5: space left 
  % #6: space right
  % #7: math style 
  \sbox0{$\ifx#7\scriptstyle\scriptscriptstyle\else\scriptstyle\fi#5#1#6\m@th$}%
  \sbox2{$\ifx#7\scriptstyle\scriptscriptstyle\else\scriptstyle\fi#5#2#6\m@th$}%
  \sbox4{$#7\dabar@\m@th$}%
  \dimen@=\wd0 %
  \ifdim\wd2 >\dimen@
    \dimen@=\wd2 %   
  \fi
  \count@=2 %
  \def\da@bars{\dabar@\dabar@}%
  \@whiledim\count@\wd4<\dimen@\do{%
    \advance\count@\@ne
    \expandafter\def\expandafter\da@bars\expandafter{%
      \da@bars
      \dabar@ 
    }%
  }%  
  \mathrel{#3}%
  \mathrel{%   
    \mathop{\da@bars}\limits
    \ifx\\#1\\%
    \else
      _{\copy0}%
    \fi
    \ifx\\#2\\%
    \else
      ^{\copy2}%
    \fi
  }%   
  \mathrel{#4}%
}
\makeatother

% for todos
\usepackage{xargs}                      % Use more than one optional parameter in a new commands
\usepackage[pdftex,dvipsnames]{xcolor}
%todos -- remove at end
\usepackage[colorinlistoftodos,prependcaption,textsize=tiny]{todonotes}
\newcommand{\unsure}[2][1=]{\todo[linecolor=red,backgroundcolor=red!25,bordercolor=red,#1]{#2}}
\newcommand{\change}[2][1=]{\todo[linecolor=blue,backgroundcolor=blue!25,bordercolor=blue,#1]{#2}}
\newcommand{\info}[2][1=]{\todo[linecolor=OliveGreen,backgroundcolor=OliveGreen!25,bordercolor=OliveGreen,#1]{#2}}
\newcommand{\improvement}[2][1=]{\todo[linecolor=Plum,backgroundcolor=Plum!25,bordercolor=Plum,#1]{#2}}



\title{Predicting Electron Paths}

% The \author macro works with any number of authors. There are two
% commands used to separate the names and addresses of multiple
% authors: \And and \AND.
%
% Using \And between authors leaves it to LaTeX to determine where to
% break the lines. Using \AND forces a line break at that point. So,
% if LaTeX puts 3 of 4 authors names on the first line, and the last
% on the second line, try using \AND instead of \And before the third
% author name.

\author{
  David S.~Hippocampus\thanks{Use footnote for providing further
    information about author (webpage, alternative
    address)---\emph{not} for acknowledging funding agencies.} \\
  Department of Computer Science\\
  Cranberry-Lemon University\\
  Pittsburgh, PA 15213 \\
  \texttt{hippo@cs.cranberry-lemon.edu} \\
  %% examples of more authors
  %% \And
  %% Coauthor \\
  %% Affiliation \\
  %% Address \\
  %% \texttt{email} \\
  %% \AND
  %% Coauthor \\
  %% Affiliation \\
  %% Address \\
  %% \texttt{email} \\
  %% \And
  %% Coauthor \\
  %% Affiliation \\
  %% Address \\
  %% \texttt{email} \\
  %% \And
  %% Coauthor \\
  %% Affiliation \\
  %% Address \\
  %% \texttt{email} \\
}

\newcommand{\xb}{\mathbf{x}}
\newcommand{\Xc}{\mathcal{X}}
\newcommand{\Zc}{{\mathcal{Z}}}
\newcommand{\Mc}{{\mathcal{M}}}
\newcommand{\Bc}{{\mathcal{B}}}
\newcommand{\Ac}{{\mathcal{A}}}
\newcommand{\Pc}{{\mathcal{P}}}
\newcommand{\bb}{{\mathbf{b}}}
\newcommand{\ab}{{\mathbf{a}}}
\newcommand{\mb}{{\mathbf{m}}}
\newcommand{\Mb}{{\mathbf{M}}}
\newcommand{\Pb}{{\mathbf{P}}}
\newcommand{\Hb}{{\mathbf{H}}}
\newcommand{\Ab}{{\mathbf{A}}}
\newcommand{\delb}{{\boldmath{\delta}}}



% The model definitions
\newcommand{\electronPath}{\Pc}
\newcommand{\moleculeSet}{\Mc}
\newcommand{\initialAndReactants}{\Mc_0, \Mc_r}


% Then the modules!
\newcommand{\fEmbed}{g_{\Ac}}
\newcommand{\fAdd}{f_{\textrm{add}}}
\newcommand{\fRemove}{f_{\textrm{remove}}}
\newcommand{\fInitial}{f_{\textrm{initial}}}
\newcommand{\fStop}{f_{\textrm{stop}}}
\newcommand{\fReagEmbed}{f_{\textrm{reagent}}}
\newcommand{\fModules}{\fEmbed, \fAdd, \fRemove, \fInitial,\fStop, \fReagEmbed}
\newcommand{\fui}{f_i}
\newcommand{\fuj}{f_j}
\newcommand{\fuk}{f_k}

\newcommand{\actionProb}[2][]{ p(a_{#2} \mid \moleculeSet_{\electronPath_{0:#2-1}^{#1}}, a^{#1}_{#2-1}, #2)}
\newcommand{\continueProb}[2]{p(s_{#1}' \mid \moleculeSet_{#2}) }

\begin{document}
% \nipsfinalcopy is no longer used

\maketitle

\begin{abstract}
The vast majority of elementary chemical reactions can be described as the movement of pairs of electrons through a set of reactant molecules. As such, reactions are often described using `arrow-pushing' diagrams which show this movement as a sequence of arrows. We propose to learn this sequence directly. Instead of predicting product molecules directly from reactant molecules in one-shot, learning a model of electron movement has the benefits of (a) being easy for chemists to interpret, (b) being able to incorporate the constraints of chemistry such as balanced atom counts each side , and (c) naturally encoding the sparsity of chemical reactions, which usually involve only a small number of atoms in the reactants.
% also it means the sides balance
\end{abstract}

\section{Background}
In this section we describe the types of reactions we consider in this paper, and how it relates to previous work on reaction prediction. We then describe recent related work on deep graph models that inspire our model in the following section.

\paragraph{Chemical reactions.}
Molecules consist of a set of atoms that are arranged into a structure by a set of bonds. As such, molecules can be depicted as a graph structure, where each node is an atom and each edge is a bond, as shown in Figure (note the convention that vertices which do not explicitly specify the atom name are assumed to be carbon C atoms). 

In reality, the structure of a molecule is due to how electrons on each atom are interacting with each other. Each single bond represents the fact that two electrons are shared between the atoms that the bond connects\footnote{The vast majority of bonds in molecules are like this (so-called \emph{covalent} bonds), although our model also accommodates ionic bonds (in which one atom completely borrows the electrons of another atom and the atoms become charged ions.}.

%Molecules are broken and built via reactions. 
Just as electrons describe the current structure of molecules, they also describe how molecules react with other molecules to produce new ones. All chemical reactions involve the movement of electrons along paths of atoms in a set of reactant molecules. This movement causes the formation and breaking of chemical bonds that changes the reactants into a new set of product molecules.\footnote{This movement happens because it allows the set of molecules to move to a lower (and thus more favorable) energy state.} In this work, we will consider reactions that satisfy the following assumptions:
% \begin{enumerate}
% \item are single-step, so-called \emph{elementary} reactions.
% \item involve a pair of electrons, so-called \emph{heterolytic} reactions.
% \item either start with electrons on single atom, or with the electrons in an existing bond.
% \end{enumerate}
\begin{assumption}
Reactions are single-step, so-called \emph{elementary} reactions.
\label{assume:elem}
\end{assumption}

\begin{assumption}
Reactions involve pairs of electrons moving, so-called \emph{heterolytic} reactions.
\label{assume:het}
\end{assumption}

\begin{assumption}
Reactions either start with electrons on single atom (called \emph{free electrons}), or with the electrons in an existing bond.
\label{assume:atom_bond}
\end{assumption}

These sorts of reactions describes the vast majority of \emph{organic reactions} (i.e., reactions involving Carbon atoms) that have a large number of applications from drug design to the invention of new materials\footnote{Organic reactions that do not satisfy these assumptions are homolytic reactions, and concerted reactions.}. For this reason organic reactions has been the focus of recent work in reaction prediction in machine learning \cite{jin2017predicting,schwaller2017found}. Note that reactions which are multi-step can be decomposed into multiple single-step reactions in order to satisfy Assumption~\ref{assume:elem}.


\paragraph{Reactions as single electron paths.}
If reactions satisfy the above assumptions, then a chemical reaction is the result of pairs of electrons moving in a \emph{single path} through the reactant atoms. Further, this electron path will alternately remove existing bonds in molecules, and form new ones. We show this alternating structure in two example single-path reactions in Figure~\ref{fig:example}. In Figure~\ref{fig:example}(a) the reaction starts with the electrons in a bond, and in Figure~\ref{fig:example}(b) the reaction starts with the electrons in an atom. 

There are a number of benefits of predicting electron paths over predicting the outcomes of reactions directly (as in previous work \cite{jin2017predicting,schwaller2017found}):
\begin{itemize}
\item \textbf{Easy to interpret}: If the model makes a mistake, it is easy to see where it goes wrong by comparing the steps of the path with the correct steps.
\item \textbf{Sparse}: Reactions often only affect between 3 and 7 atoms out of anywhere from 10-50 reactant atoms. Modeling the reaction as a path allows us to exploit this sparsity.
\item \textbf{Chemical constraints}: Learning a path allows us to easily incorporate chemical constraints, such as the alternating removal and addition of bonds, among others.
\end{itemize}

\paragraph{Formal notation.}
Below is the formal notation we will use to build a model that describes reactions as electron paths:
\begin{itemize}
\item A maximum number of path timesteps $T$.
\item An initial set of reactant molecules $\Mc_0$, this is a set of molecular graphs with vertices called atoms $\Ac$ and edges called bonds $\Bc$. %We also assume we have a continuous representation of the set of molecules: $\mb_1$, of atoms $\ab$, and bonds $\bb$.
\item A final set of product molecules $\Mc_T$. %, and its continuous representation: $\mb_T$. 
Note, that if we believe it requires less than $T$ steps to go from the reactants to the products we assume we can perform a 'null' step in which the electrons do not move, thus the molecules do not change.
\item A reaction is just a sequence of electron movements from atom to atom. Formally, we write an electron path as an ordering of single atoms $\Pc = (a_0, a_1, \ldots, a_{T-1})$. %, with continuous representation $\Pb = [\ab_1, \ldots, \ab_{T-1}]$.
\end{itemize}





\begin{figure*}
\centering
\includegraphics[width=\textwidth]{rxn_example}
\vspace{-3ex}
\caption{Two example reactions and their electron paths. In reaction (a), the reaction starts from an existing bond between Lithium (atom 1) and Carbon (atom 2). In reaction (b) the reaction starts from a lone-pair of electrons on Nitrogen (atom 1), which we represent formally as a bond with itself.}
\label{fig:example}
\end{figure*}

% KEY: separate assumptions from data processing
% !TEX root =  ../main_iclr.tex



In this section we define a probabilistic model that describes the movement of electrons that define linear topology reactions.
We represent a set of molecules as a set of graphs $\moleculeSet$, with atoms $\Ac$ as vertices and bonds $\Bc$ as edges;
each connected component of the graph defines an individual molecule.
We can associate an ordering over all the atoms in all the molecules in the set using an {\em atom map} number:
an integer label assigned to each non-hydrogen atom in both the reactants and the products which 
both permits easy matching between atoms before and after the reaction, and
gives us a consistent way to index particular atoms. Molecules input into the model are first put in a Kekul\'e form, a process which makes explicit the location of single and double bonds in aromatic structures;
each bond $b \in \Bc$ is either a single, double, or triple bond.

Each atom $v \in \Ac$ includes a set of features, such as its atom type (e.g. carbon, oxygen, \dots); the full list of input atom features can be found in Table 3 of the appendix.
 However, when learning functions that operate over these atoms we do not work with these raw features directly but instead with the atom, or equivalently node, embedding. 
 These contain information about the atom of interest as well as its surrounding neighborhood.
 These can be computed using any model that is able to compute graph-isomorphic features, for instance usually via message-passing techniques \citep{gilmer2017neural}; we choose to use 4 layer gated graph neural network (GGNN) message functions \citep{li2016gated}, for which we include a short review in the appendix. 
 These produce a $d$-dimensional embedding of an atom and if we stack these vectors up as rows (using the atom mapping to define an order)  we have the matrix $\nodeEmbeddings{\moleculeSet} \subseteq \mathbb{R}^{|\Ac|\times d}$, containing all the node embeddings for a particular molecule set, $\moleculeSet$.

Sometimes we wish to agglomerate the node embeddings belonging to a set of nodes $\Ac'$ (with $\Ac' \subseteq  \Ac$), to form a graph embedding \citet[\S B.1]{li2018learning}, that is a vector that represents multiple nodes but that is invariant to any particular node ordering. 
These functions have also previously been referred to as aggregation graph transformations \citep[\S3]{Johnson2017-pd}.
We denote these functions that map from $\nodeEmbeddings{\moleculeSet'}$ to a $q$-dimensional vector, $r: \mathbb{R}^{|\Ac'|\times d} \to \mathbb{R}^q$.
They obtain their node order invariance by performing a weighted sum over the nodes.


Given an initial set of reactant molecules $\moleculeSet_0$ and a set of reagent molecules $\moleculeSet_r$, 
our model defines a conditional distribution over a sequence of atoms (which we also refer to as actions) $\electronPath_{0:T} = (a_0, a_1, \ldots, a_T)$,
which fully characterizes the electron path.
This electron path in turn deterministically defines both a final product $\moleculeSet_{T+1}$, 
denoting the outcome of the reaction,
as well as a sequence of intermediate products $\moleculeSet_t$, for $t = 1,\dots,T$,
which correspond to the state of the graph after the first $t$ steps in the subsequence $\electronPath_{0:t} = (a_0, \dots, a_t)$ are applied to the initial $\moleculeSet_0$. We also define a stopping sequence $\mathcal{S}_t = (s_0, \ldots, s_{T+1})$ which indicates if the reaction should stop (i.e, $s_t\!=\!1$ if the reaction should stop and is $0$ otherwise). 

We propose to learn a distribution $p_\theta( \electronPath_{0:T} \mid \moleculeSet_0, \moleculeSet_r)$ over electron movements. 
We first detail the generative process %(i.e., the forward pass) 
that specifies $p_\theta$, before describing how to train the model's parameters, $\theta$.


\subsection{Generative process}


\begin{figure*}
\centering
\includegraphics[width=\textwidth]{reaction_model_blue}
\caption{
 This figure shows the sequence of actions in transforming the reactants in box 1 to the products in box 9.
 The sequence of actions will result in a sequence of pairs of atoms, between which bonds will alternately be removed and created, creating a series of intermediate products. 
At each step the model sees the current intermediate product graph (shown in the boxes) as well as the previous action, if applicable, shown by the grey circle. It uses this to decide on the next action.
We represent the characteristic probabilities the model may have over these next actions as colored circles over each atom.
Some actions are disallowed on certain steps, for instance you cannot remove a bond that does not exist; these blocked actions are shown as red crosses.
}
\label{fig:reaction_model}
\end{figure*}







First note that since our reactions are a single path of electrons through the reactants then, at any point, the next step in the path depends only on (i) the intermediate molecule formed by the action path up to that point, (ii) the previous action taken (indicating where the free pair of electrons are) and (iii) the point of time through the path, indicating whether we are on an add or remove bond step. 
We make the simplifying assumption that the stop probability and the actions after the initial action $a_0$ do not depend on the reagents. This leads to a parameterized model with dependency structure:
\begin{align}
\label{eq:jointprob}
p_\theta(\electronPath_{0:T} \mid \moleculeSet_0, \moleculeSet_r) 
&=
	p_\theta(s'_0 \mid \moleculeSet_0)
	p_\theta(a_0 \mid \moleculeSet_0, \moleculeSet_r)\\ \nonumber &\quad \times
	\left[\prod_{t=1}^{T}
		p_\theta(s_{t}' \mid \moleculeSet_{t})
		p_\theta(a_t \mid \moleculeSet_{t}, a_{t-1}, t)
	\right]
	p_\theta(s_{T+1} \mid \moleculeSet_{T+1})
	,
\end{align}
where we have defined $p_\theta(s'_t \mid \moleculeSet_t) \equiv 1 - p_\theta(s_t \mid \moleculeSet_t)$ to be the probability of {\em continuing} a reaction given the current molecule set $\moleculeSet_t$.
The other terms include $p_\theta(a_0 \mid \initialAndReactants)$, the probability of the initial state $a_0$ given the reactants and reagents; 
the conditional probability $p_\theta(a_t \mid  \moleculeSet_t, a_{t-1}, t)$ 
%\todo[]{maybe somehow refer to "bond type" instead of $t$?} 
of next state $a_t$ given the intermediate products $\moleculeSet_t$ for $t > 0$;
and the probability $p_\theta(s_t \mid \moleculeSet_t)$ that the reaction terminates with final product $\moleculeSet_{t}$.


\begin{algorithm}[t]
  \caption{The generative steps of ELECTRO.}
  {\bf Input:}~~Reactant molecules $\Mc_0$ (consisting of atoms $\Ac$), reagents $\Mc_r$, atom embedding function $\fEmbed(\cdot)$, graph embedding function $r(\cdot)$, time steps $T^\mathrm{max}$
  
  \begin{algorithmic}[1]
  	\STATE $\nodeEmbeddings{\moleculeSet_0} = \fEmbed(\Mc_0)$ \COMMENT{atom embedding of reactants}
  	\STATE $\contextVect_\mathrm{reagent} = r(\Mc_r)$ \COMMENT{graph embedding of reagents}
  	%\STATE $\actionLogits = $
  	\STATE $p_{\mathrm{start}}(a_t \mid \moleculeSet_0, \moleculeSet_r) = \mbox{softmax}(\fInitial(\nodeEmbeddings{\moleculeSet_0}, \contextVect_\mathrm{reagent}))$ \COMMENT{probability $a_t$ starts the reaction}
  	\STATE $a_{0} \sim p_{\mathrm{start}}(a_t \mid \moleculeSet_0, \moleculeSet_r)$
  	\FOR{$t = 1, \ldots, T^\mathrm{max}$}
  		\STATE $\moleculeSet_t \leftarrow \moleculeSet_{t-1}, a_{t-1}, \ldots, a_{t-1}$ \COMMENT{modify molecules based on previous molecule and action}
  		\IF{$t \mod 2 = 1$}
  			\STATE a
  		\ENDIF
  		\STATE a
  		%\STATE \mathbf{h}_
  	\ENDFOR
 %  	% Set up pool of completed paths to sort later
 %  	\STATE $\outputPool = \{\left( \emptyset, \log (1 - \cProbCont(\Mc_0)) \right) \}$  \COMMENT{This set will store all completed paths.}
 %  	\STATE $\removeFlag \!=\!1$ \COMMENT{Remove flag}
  	
 %  	% Pick the first action.\\
 %  	\STATE
	% \STATE $\hat{\Bc} = \emptyset$.  \COMMENT{This set will store all possible open paths. Cleared at start of each timestep.}	
	% \FORALL{$v \in \Ac$} 
	% 	\STATE $ \cPath = (v)$
	% 	\STATE $ \lProb = \log \cProbCont(\Mc_0) + \log \cProbInitial(v, \moleculeSet_0, \moleculeSet_r)$
	% 	\STATE $\hat{\Bc} = \hat{\Bc} \cup \{\left(\cPath, \lProb \right)\}$
	% \ENDFOR
	% \STATE  $\Bc_{0} = \texttt{pick\_topK\_actions}(\hat{\Bc})$ \COMMENT{We filter down to the top K most promising actions.}
	
	% % Then we evaluate the next stages.
	% \STATE
	% \FOR{t in $(1, \ldots, T^\mathrm{max})$}
	% 	\STATE $\hat{\Bc} = \emptyset $ 
				
	% 	% We take all the previous top K open paths from the previous step...
	% 	\FORALL{$(\cPath, \lProb) \in \Bc_{t-1}$} 
			
	% 		% We evaluate their stop probability at that point and add these stopped version to the completed pool.
	% 		\STATE $\Mc_\cPath = \texttt{calc\_intermediate\_mol}(\Mc_0, \cPath)$
	% 		\STATE $p_c = \cProbCont(\Mc_\cPath)$
	% 		\STATE $\hat{\Pc} = \hat{\Pc} \cup \{(\cPath, \lProb + \log (1 - p_c))\}$
			
	% 		% We then see what would happen if we continued and picked another action.
	% 		\FORALL{$v \in \Ac$}
	% 			\STATE $\cPath' = \cPath^\frown (v)$ \COMMENT{New proposed path is concatenation of old path with new node.}

	% 			\STATE $v_{t-1} = $ last element of $\cPath$
	% 			\STATE $\hat{\Bc} = \hat{\Bc} \cup \{(\cPath' , \lProb + \log p_c + \log \cProbAct(v, \Mc_\cPath, v_{t-1}, \removeFlag) )\}$
	% 		\ENDFOR
	% 	\ENDFOR
		
	% 	% We next prune down the search space for next iteration to our beam width.
	% 	\STATE  $\Bc_{t} = \texttt{pick\_topK\_actions}(\hat{\Bc})$ 
		
	% 	% We indicate that the next step will be opposite step:
	% 	\STATE $\removeFlag = \removeFlag + 1 \mod 2$. \COMMENT{If on add step change to remove and vice versa.}
	% \ENDFOR
	
 %  % Finally we sort the pool and we are done
 %  \STATE
 %  \STATE $\outputPool = \texttt{sort\_on\_prob}(\outputPool)$
  \end{algorithmic}
  {\bf Output:}~~Valid completed paths and their respective probabilities, sorted by the latter, 
  \label{algo:valid_path}
\end{algorithm}


It is possible to stop prior to selecting a first atom $a_0$, indicating that no reaction would take place.
However, we restrict our model to not stop at step $t\!=\!1$, as it is necessary to pick up a complete electron pair. 
Given any particular selected atom $a_t$ which extends the reaction path, we can deterministically update the previous molecular graph $\moleculeSet_{t}$ to produce the next set of (intermediate) products $\moleculeSet_{t+1}$.

Given our reaction assumptions, then, as stated earlier, there are two types of electron movements that alternate: 
(i) movement that \emph{removes an existing bond}, and 
(ii) movement that \emph{adds a new bond}. 
We define atoms with free electrons as having a self-bond.
Thus, all reactions start by first selecting an atom, removing a bond (between two different atoms, or a self-bond), and then alternately adding and removing bonds;
we can determine whether a particular step is an add step or a remove step by inspecting $t$.
Note that $\moleculeSet_1 = \moleculeSet_0$, as the initial action of selecting $a_0$ does not form or remove any bonds.
Figure~\ref{fig:reaction_model} presents a simple example reaction which demonstrates all the critical features of the model;
the subfigures show the sequence of intermediate products and the distributions over actions.



We are left now with defining the functional form of our conditional distributions in  Eq.~\eqref{eq:jointprob} for continuing $p_\theta(s'_t \mid \moleculeSet_t)$, picking the initial action $p_\theta(a_0 \mid \initialAndReactants)$, and picking subsequent actions $p_\theta(a_t~|~\moleculeSet_t, a_{t-1}, t)$, all of which are parameterized by neural networks.
At each step of the electron path, the network takes the current intermediate graphs, 
the previous action, and the reagents if relevant, 
and computes a probability distribution over next possible actions (i.e., selecting a particular atom, or stopping).
The structure of these networks is described in the following section, although we defer full architectural details (e.g.\ number of layers and hidden units) 
and training settings to the appendix.








\subsection{Computing probabilities over actions}

We now define each of our parameterized distributions over actions.
The simplest of these is $p_\theta(s'_t \mid \moleculeSet_t)$, which is the probability of continuing given the set of intermediate products at time $t$. 
This probability is computed from a graph embedding via the function $\fEmbedGraphs_{\textrm{stop}}$, which projects down to a single dimension. We then map this to the $[0,1]$ interval via the sigmoid function $\sigma$ which yields the overall expression
\begin{align}
%$
p_\theta(s'_t \mid \moleculeSet_t) = \sigma(\fEmbedGraphs_{\textrm{stop}}(\nodeEmbeddings{\moleculeSet_t})).
%$
\end{align}
Each of the three parameterized conditional probability distributions for the {\em start}, {\em add} and {\em remove} steps have similar forms, each defining a probability vector over actions.
The transition distribution $p_\theta(a_t \mid  \moleculeSet_t, a_{t-1}, t)$ 
which selects the next atom in the sequence $\electronPath$
can be split into two distributions depending on the parity of $t$:
the remove bond step distribution $p_\theta^\textrm{remove}(a_t \mid  \moleculeSet_t, a_{t-1})$ is used when $t$ is odd, 
and the add bond step distribution $p_\theta^\textrm{add}(a_t \mid \moleculeSet_t, a_{t-1})$ is used when $t$ is even. 

These three modules each have the same overall functional form
\begin{align}
\actionLogits &= f(\nodeEmbeddings{\moleculeSet_t}, \contextVect), \\
p_\theta(a_t \mid \cdots) &\propto \bm{\beta} \odot \mbox{softmax}(\bm{\actionLogits})
\end{align}
where $f$ is one of the networks $\fInitial, \fAdd$, or $\fRemove$; 
$\contextVect$ is a context vector, and $\bm{\beta}$ is a binary mask.

Each of the three actions has a different context and mask.
The add step $p_\theta^\textrm{add}(a_t \mid \moleculeSet_t, a_{t-1})$ and remove step $p_\theta^\textrm{remove}(a_t \mid \moleculeSet_t, a_{t-1})$,
 have as context the node embedding of the atom selected at the previous step, $\contextVect_{a_{t-1}} = \nodeEmbeddings{\moleculeSet_t, a_{t-1}}$. 
For the initial step, this context vector $\contextVect_\mathrm{reagent}$ is an embedding of all the reagents present, computed 
by a graph embedding function $\fEmbedGraphs_\mathrm{reagent}$.
When computing the output probabilities,
we use the binary vector $\bm{\beta}$ to mask out specific actions known to be impossible.
The value of this differs for the {\em start}, {\em add} and {\em remove} steps;
for the start step any action can be picked, so $\bm{\beta}$ is the all-ones vector.
For the remove step, $\bm{\beta}_\mathrm{remove}$ masks out (i.e.\ is set to zero for) any bonds that do not currently exist and thus cannot be removed (noting though that self-bonds are permitted in the first remove step).
For the add step, $\bm{\beta}_\textrm{add}$ only masks out the previous action, preventing the model from stalling in the same state for multiple time-steps. 
\paragraph{Training}
We can learn the parameters $\theta$ of all the parameterized functions, including those producing node embeddings, by maximizing the log likelihood of a full path $\log p_\theta(\electronPath_{0:T} \mid \moleculeSet_0, \moleculeSet_r)$.
This is evaluated by using a known electron path $a_t^\star$ and intermediate products $\moleculeSet_t^\star$ extracted from training data,
rather than on simulated values. 
This allows us to train on all stages of the reaction at once, given electron path data.
We train our models using Adam \citep{kingma2014adam} and an initial learning rate of $10^{-4}$,
with minibatches of size one, where minibatches often consist of multiple intermediate graphs.

\paragraph{Prediction}
Once trained, we can use our model to sample chemically-valid paths given an input set of reactants $\moleculeSet_0$ and reagents $\moleculeSet_r$, 
simply by simulating from the conditional distributions until sampling a stop value $s_t$.
We instead would like to find a ranked list of the top-$K$ predicted paths, and do so using a modified beam search,
in which we roll out a beam of width $K$ until a maximum path length $T^\mathrm{max}$,
while recording all paths which have terminated.
This search procedure is described in detail in Algorithm 1 in the Appendix.








 




% While this assigns probabilities to discrete actions this can be a function of continuous embeddings of molecules as above. 

% To learn $g_\theta$, one idea is to sample a path $\Pc$ and apply it to our initial set of molecules $\Mc_0$. Using the known deterministic function $f$ we can produce a final predicted set of molecules $\hat{\Mc}_T$. However, even if we have a loss function $\ell(\hat{\Mc}_T,\Mc_T)$ we cannot use REBAR or RELAX to compute $\frac{\partial \ell(\hat{\Mc}_T,\Mc_T)}{\partial \theta}$ because $f$ is not stochastic. Maybe it's possible to make $f$ stochastic, then maybe it is possible to apply REBAR or RELAX.

% Another idea is to learn $g_\theta$ via maximum likelihood. Specifically, $g_\theta$ assigns some probability to all possible paths, so we can adjust $\theta$ to make the paths that lead to $\Mc_T$ more likely and those that do not less likely. I believe an efficient way to do this is to roll out only a few paths to produce $\hat{\Mc}_T$ (i.e., using the valid path sampling method using Algorithm~\ref{algo:valid_path}) and then update $\theta$ via $\frac{\partial \ell(\hat{\Mc}_T,\Mc_T)}{\partial \theta}$. Maybe Monte Carlo Tree Search could be useful here? I need to read more about this. 

% \subsection{Notes}
% In general I haven't given as much thought to a stochastic model. This is because ultimately we don't really want to give a chemist a distribution over electron paths, they want to know an actual prediction of electron movements. One could argue that this distribution might be a proxy for reaction `energy', but I don't think we need to learn a distribution to get this. We could make Algorithm 1 stochastic by instead of selecting the closest atom in steps 6 and 12, we select an atom with Gaussian probability given by the Euclidean distance between atoms in $\Mc_t$ and the predicted action $\hat{\ab}$. It's not clear to me whether this is a better or worse proxy for reaction energy.

% The main question is whether it will be easier to learn a stochastic function or two deterministic functions that give realistic electron paths. I think this boils down to (a) can we sample enough roll-outs to get a peaky distribution, (b) will learning both $g$ and $f$ lead to too much approximation error.



% We propose to learn a function $g: \Mc \rightarrow \Pc$. 

% Because we do not know the true path $\Pc$, we can only receive a learning signal from the final predicted product molecules $\hat{\Mc}_T$ (compared to the true final molecules $\Mc_T$. This final predicted product is a deterministic (known) function $f$ of the initial molecules $\Mc_0$ and predicted actions $\hat{\Pc}$ (from $g$), as such $f(\Mc_0, \hat{\Pc}) = \hat{\Mc}_2, \ldots, \hat{\Mc}_T$.

% Ideally, we would like to learn the parameters of $g$, called $\theta$, to minimize the difference between the predicted final molecules $\hat{\Mc}_T$ and $\Mc_T$ (i.e., via some loss function $\ell$). However, we cannot resort to gradient-based techniques to learn $g$ because the inputs and outputs of $g$ are discrete objects, and the function $f$ producing $\hat{\Mc}_T$ is also discrete so the gradient $\frac{\partial \ell(\hat{\Mc}_T,\Mc_T)}{\partial \theta}$ does not exist. To solve this, we propose two types of models for this problem.

% \paragraph{Deterministic.}
% Instead, we assume we have continuous mappings from molecules $\Mc$ to vectors $\mb$ and paths $\Pc$ to matrices $\Pb$. We propose to learn continuous functions $g_\theta: \mb \rightarrow \Pb$ and $f_\phi: \mb, \Pb \rightarrow \mb, \ldots, \mb$. Then we can compute derivatives of a loss function $\ell(\hat{\mb}_T,\mb_T)$ with respect to parameters $\theta$ as follows. Let $\hat{\mb}_T = [f(\mb_1,g(\mb_1))]_T$. Then our gradient is $\frac{\partial \ell(\hat{\mb}_T,\mb_T)}{\partial \theta} = \frac{\partial \ell(\hat{\mb}_T,\mb_T)}{\partial f}\frac{\partial f}{\partial g}\frac{\partial g}{\partial \theta}$.

% Note that now, $f_\phi$ is unknown. So we propose to learn it given observed traces: $(\mb_1,\Pb,\mb_2,\ldots,\mb_T)$. For simplicity define $\Mb_{2:T} = [\mb_2,\ldots,\mb_T]$, and $f(\mb_1,\Pb) = \hat{\Mb}_{2:T}$. Then, given a loss function $l(\hat{\Mb}_{2:T},\Mb_{2:T})$ we can learn the parameters of $f$, called $\phi$ via the gradient $\frac{\partial l(\hat{\Mb}_{2:T},\Mb_{2:T})}{\partial \phi} = \frac{\partial l(\hat{\Mb}_{2:T},\Mb_{2:T})}{\partial f} \frac{\partial f}{\partial \phi}$.

% In order to ensure that $g_\theta$ produces paths $\hat{\Pb}$ that are valid (i.e., that alternately remove and add bonds) we propose to take a predicted path $\hat{\Pb}$ and map it to a valid path $\Pb^*$ as described in Algorithm~\ref{algo:valid_path}. Given a valid path, we propose a loss function $L(\hat{\Pb},\Pb^*)$ and learn $g_\theta$ to minimize this loss via $\frac{\partial L(\hat{\Pb},\Pb^*)}{\partial \theta} = \frac{\partial L(\hat{\Pb},\Pb^*)}{\partial g} \frac{\partial g}{\partial \theta}$.



% %another loss function 
% %Finally, one last appealing property of this model is that if similar molecules have similar continuous representations embeddings then the functions $g_\theta, f_\phi$ allow us to generalize to similar molecules we have not seen before.
% Function $f_\phi$ could be an RNN and $g_\theta$ could be a fully connected network. We could consider an alternative $g_\theta$ that is conditioned on previous states which could be an RNN. 



% \paragraph{Stochastic.}

%Given a learned distribution we could sample valid paths using Algorithm~\ref{algo:valid_path}. We could use a model similar to DRAW. Maybe it is possible to not learn $f$ and instead use a dynamic programming algorithm to update $g_\theta$. Otherwise, we could learn $f$ as a stochastic function over discrete states. In general I haven't given as much thought to a stochastic model.
%Note that this distribution is not Markov as we need to know if the previous two electron movement added or removed a bond, in order to determine whether we need to consider only existing bonds or not. We could use a model similar to DRAW and mask all samples so agree with these constraints.




% TODO:
% - write intuitive idea
% - formalize variables
% - wait a long time until writing objective
% - wait even longer to write optimization

\bibliography{bibliography}
\bibliographystyle{plainnat}


\end{document}
