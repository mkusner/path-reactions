\documentclass{article}

\usepackage{nips_2018_author_response}

\usepackage[utf8]{inputenc} % allow utf-8 input
\usepackage[T1]{fontenc}    % use 8-bit T1 fonts
\usepackage{hyperref}       % hyperlinks
\usepackage{url}            % simple URL typesetting
\usepackage{booktabs}       % professional-quality tables
\usepackage{amsfonts}       % blackboard math symbols
\usepackage{nicefrac}       % compact symbols for 1/2, etc.
\usepackage{microtype}      % microtypography

\usepackage[colorinlistoftodos,prependcaption,textsize=tiny]{todonotes}


\begin{document}
Thank you for taking the time to review our paper.

\todo[]{started chucking experimental results down but need to work out how to phrase and present them}



The timings of the models are given in Table \ref{table:timings}. We shall include these in the paper.
 For Electro we use a beam width of ten. 
 Although the overall time is greater than the others, only 0.044 secs is used to compute the probabilities and the majority of the time (0.193 secs) is used using RDKIT to create intermediate molecules or derive features from them, an operation that is not currently parallelized across the different beams. 
 Instead as the whole predict operation for Electro only takes 0.428 secs on a CPU we take advantage of the embarrassingly parallel nature of the task across different reaction predictions when running Electro on the test set.
 If we are interested in only computing the log likelihood of a reaction (and so have access to all the intermediate steps), this takes 0.007 secs on a GPU for Electro.
 
 \todo[]{Can also add the fact we have less parameters.}
 
 

\begin{table}[h]
  \caption{Time taken in seconds to predict product from the initial reactant and reagent molecules.
  These have been run on a NVIDIA 1080 GPU, with the mean time taken over the first 500 items of the test set.
  * As no code is available for [14] we instead report the timings given in Section 6.2 of their paper.
  }
  \label{table:timings}
  \centering
  \begin{tabular}{llll}
    \toprule
    & Electro (Ours) & WLDN [5] & Seq2Seq [14]  \\
    \midrule
    Timing (secs) & 0.337   & 0.034 &  0.025*     \\
    \bottomrule
  \end{tabular}
\end{table}
\todo[inline]{Check wldn timings as they give 50ms in paper. Maybe they include the initial rdkit time.}


The motivation of only including reagent context at the start is that this is often the most challenging decision, and that action steps after this have access to the previous atom as context, making it an easier task.
Qualitatively, when running Electro-lite on the separate validation set we would see that the model often had the most errors on the first step, and that after picking this first step the next stages would often be correctly predicted.
One of the nice features of the approach is that we can easily break down the loss (or equivalently the negative log likelihood) of each prediction our model makes. 
Empirically we found that Elctro-lite on the first 500 items of the validation set obtained a median loss of 0.209 on predicting where to start, whereas the first add step (when applicable) would contribute a median loss of 0.002, highlighting the sometimes much poorer performance of this module.
When testing Electro, its initial select module would get a median loss of 0.023, showing that the addition of reagents, greatly improves the prediction of this task.
\todo[]{If manage to get the experiment tell the performance of the model where reagents get added everywhere}

\end{document}
